\section{Energy Shaping}
\showtoc

\subsection{Energy Shaping with Control Lyapunov Functions}

\begin{frame}[t]
  \frametitle{Motivation}
  \begin{block}{Main Question}
    Can we use an understanding of energy exchange to improve robustness  of
    periodic orbits in hybrid mechanical systems?
  \end{block}

  \begin{block}{Observations}
    \begin{itemize}
    \item Numerous control design schemes exist for stabilizing hybrid mechanical
      systems to periodic orbits.
    \item Some controllers produce good behavior locally but lack robustness.
    \item Periodic orbits have associated energy functions with level sets which
      are invariant under the orbits.
    \end{itemize}
  \end{block}
\end{frame}

\begin{frame}[t]
  \frametitle{Overview}
  \begin{block}{Main Idea}
    Add robustness to a periodic behavior by imposing convergence on a conserved
    energy function, $\Ec : \x \to \R$, to a level set which is known to be
    invariant under the system dynamics.
  \end{block}
  
  \begin{block}{Control Objective}
    Choose control input $\mu\arx$ such that $\| \mu\arx \|$ is minimized and
    $\Ec\arxt \to \Eref$ as $t \to \infty$.
  \end{block}

  \begin{block}{Exponential Convergence}
    To achieve exponential stabilization, $\Ec\arxt$ should satisfy\vspace{-.4em}
    \begin{align*}
      \Ec\arxt \leq \Ec\arxzero e^{-\beta t} \mbox{ for } t \geq 0, \beta > 0.
    \end{align*}
  \end{block}
\end{frame}

\begin{frame}[t]
  \frametitle{Conserved Energy Functions}
  \only<1> {
    \begin{block}{Conservative Systems}
      A conservative system with state space $\x = \argsqdq$ is modeled as
      \begin{align*}
        \HCSbar = \left\{
          \begin{array}{l l}
            \left.\begin{array}{r c l}
                \hspace{.58em}\dx &=& \xfbar\argsqdq + \xgbar\argsq \, \uu
              \end{array}\right\}  & \mbox{if } \argsqdq \in \D \setminus \S,\\
            \left. \begin{array}{r c l}
                \qp &=& \Deltaq\argsqm\\
                \dqp &=& \Deltadq\argsqdqm
              \end{array} \right\} & \mbox{if } \argsqdq \in \S,
          \end{array}\right.
      \end{align*}
      where $\xfbar\argsqdq := \xf\argsqdq$ and $\xgbar\argsq := \xg\argsq$ for
      notational clarity. Total energy is conserved through the continuous
      dynamics, i.e.,
      \begin{align*}
        \Ec\argsqdq := T\argsqdq + U\argsq.
      \end{align*}
    \end{block}
  }

  \only<2> {
    \begin{block}{Nonconservative Systems}
      To properly handle the flow of energy due to $\vv\argsqdq$, define a storage
      function, $\W$, which obeys the differential equation
      \begin{align*}
        d\W = \vfR\argsqdq \, dt = \left( \B\argsq \, \vv\argsqdq \right)^{T}
        \frac{d\q}{dt}\art \, dt.
      \end{align*}
      Using $\W$, augment the state space, i.e., $\x := (\q, \dq, \W)$, and the vector
      fields (subsuming $\vv\argsqdq$ under $\xfbar\argsqdq$), i.e, 
      \begin{align*}
        \xfbar\argsqdq := \left(\begin{array}{c}
            \xf\argsqdq + \xg\argsq \, \vv\argsqdq\\
            \vfR\argsqdq
          \end{array}\right), &&
        \xgbar\argsq := \left(\begin{array}{c}
            \xg\argsq\\
            \boldzero
          \end{array}\right).
      \end{align*}
    \end{block}
  }
  \only<3> {
    \begin{block}{Nonconservative Systems}
      Use the augmented state to define the hybrid control system
      \begin{align*}
        \HCSbar = \left\{
          \begin{array}{l l}
            \left.\begin{array}{r c l}
                \hspace{1.15em}\dx &=& \xfbar\argsqdq + \xgbar\argsq \, \uu
              \end{array}\right\}  & \mbox{if } \argsqdq \in \D \setminus \S,\\
            \left. \begin{array}{r c l}
                \qp &=& \Deltaq\argsqm\\
                \dqp &=& \Deltadq\argsqdqm\\
                \Wp &=& \DeltaW = 0
              \end{array} \right\} & \mbox{if } \argsqdq \in \S.
          \end{array}\right.
      \end{align*}
      For such a system, the following quantity is conserved:
      \begin{align*}
        \Ec\argsqdqW := T\argsqdq + U\argsq - W.
      \end{align*}
    \end{block}
  }
\end{frame}

\begin{frame}[t]
  \frametitle{Rapidly Exponentially Stabilizing CLFs}
  A \blue{rapidly exponentially stability control Lyapunov function (RES--CLF)}
  $\Ve : \X \to \Rnn$ satisfies
  \begin{align*}
    &c_{1} \nx^{2} \leq \Ve\arx \leq \frac{c_{2}}{\resclfparam^{2}} \nx^{2},\\
    &\inf_{\uu \in \U} \Lie{\xfbar}\Ve\arx + \Lie{\xgbar}\Ve\arx \, \uu +
    \frac{c_{3}}{\resclfparam} \Ve\arx \leq 0
  \end{align*}
  for $c_{1}, c_{2}, c_{3} > 0$ exhibits exponential convergence. If the above
  are satisfied, then it is also true that
  \begin{align*}
    \left\| \pd{\Ve\arx}{\x} \right\| \leq c_{4} \nx.
  \end{align*}
\end{frame}

\begin{frame}[t]
  \frametitle{Energy Shaping}
  Consider a conserved energy function $\Ec\arx$ on a hybrid control system
  $\HCSbar$ which has a periodic orbit $\orbit$ and define a Lyapunov candidate:
  \begin{align*}
    V\arx = \frac{1}{2} \left(\Ec\arx - \Eref\right)^{2},
  \end{align*}
  with $\Eref$ the constant energy level of the system on the orbit
  $\orbit$. For a RES--CLF, we seek a feedback control law, $\mu\arx$, such that
  \begin{align*}
    \Lie{\xfbar} V\arx + \Lie{\xgbar} \V\arx \, \mu\arx + \epsilon \V\arx &\leq 0.
  \end{align*}
\end{frame}

\begin{frame}[t]
  \frametitle{Quadratic Program Formulation}
  The linear form of the RES--CLF condition suggests
  \begin{align}
    \nonumber
    \mueps\arx = \argmin_{\uu \in \R^{n}}  \, & \uu^T \uu,\\
    \mbox{s.t. } & \Aclf\arx \, \uu \leq \bclf\arx
  \end{align}
  which encodes the dynamics of the system. This controller imposes exponential
  stabilization of the energy as defined by the RES--CLF.
\end{frame}

\begin{frame}[t]
  \frametitle{Main Theorem}
  \begin{block}{Theorem [Energy Shaping]}
    Given an exponentially-stable cycle in a hybrid system, application
    of the energy shaping controller results in the closed-loop hybrid system
    \begin{align*}
      \HS_{\resclfparam} = \left\{
        \begin{array}{r c l l}
          \dx &=& \xfbar\arx + \xgbar\arx \, \mueps\arx, & \x \in \D \setminus \S,\\
          \xp &=& \Delta(\xm), & \x \in \S,
        \end{array}\right.
    \end{align*}
    which is exponentially stable about the hybrid periodic orbit $\orbit$ for
    large enough $\resclfparam$.
  \end{block}
\end{frame}

\begin{frame}[t]
  \frametitle{Overview of Proof}
  \begin{block}{Sketch of Proof [Energy Shaping]}
    \begin{enumerate}
    \item Transform the coordinates into a more intuitive form.
    \item Define a discrete Lyapunov candidate function valid on the \Poincare{} map.
    \item Show the conditions for stability through bounding arguments.
    \end{enumerate}
  \end{block}
\end{frame}

