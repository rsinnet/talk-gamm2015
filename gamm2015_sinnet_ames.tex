% 
%%%%%%%%% Do not modify anything here inside
\documentclass[english]{article}
\usepackage[a4paper]{geometry}
\geometry{verbose,tmargin=2cm,bmargin=2cm,lmargin=2cm,rmargin=2cm}

\usepackage[T1]{fontenc}
\usepackage[latin9]{inputenc}
\usepackage{amsthm}
\usepackage{amsmath}
\usepackage{amsfonts}
\usepackage{babel}

\begin{document}
%%%%%%%%% Your contribution (Title and authors) starts from here

\title{Optimal Energy-Based Control of Hybrid Systems\\ with Applications to
  Robotic Walking}

\author{\underline{Ryan W. Sinnet} and Aaron D. Ames \\
  Texas A\&M University}%
% \\%Affiliation of author2}


%%%%%%%%% Do not modify anything here inside
\date{}
\maketitle
\thispagestyle{empty}
%%%%%%%%% Your contribution (Abstract content) starts from here

Over the last fifty years, researchers have tackled the problem of bipedal
robotic locomotion using a range of approaches with varying degrees of
formality \cite{Grizzle2014}.
%
In this presentation, we are interested in one particular approach known as {\em
  energy shaping}.
%
The main idea behind energy shaping approaches involves using the structure
of energy to create stabilizing controllers for periodic behaviors in dynamical
systems.
%
The primary focus is on recent results \cite{Sinnet2015,Sinnet2015a} which
demonstrate an optimal controller for stabilizing the energy dynamics of
periodic behaviors in hybrid mechanical system while maintaining exponential
stability of the overall hybrid systems.

%
%The principles involved rely on the well-developed analytical mechanics methods
%of Lagrange as well as the body of literature on stability which centers around
%the ideas presented by Lyapunov.
%
In order to apply energy shaping, one begins with an autonomous hybrid
mechanical system which contains a locally exponentially stable periodic orbit,
possibly induced by some feedback control law.
%
For systems of this type, a conserved energy quantity exists through the
continuous dynamics and it comprises kinetic and potential energy as well as an
additional term.
%
This additional term can be treated with a storage function which tracks the
energy flowing out of the system; such energy flow occurs due to
non-conservative forcing.
%
For periodic behaviors, this conserved quantity is also periodic and, by
resetting the storage function appropriately through the discrete dynamics, can
be made to be a constant.
%
Using the conserved quantity, which is a function of the system state, and its
constant reference level, one can design an energy shaping controller to drive
the system conserved energy of the system to the reference level, thereby
stabilizing the energy dynamics of the system.
%
The particular approach discussed achieves this energy stabilization through the
use of a {\em control Lyapunov function (CLF)} \cite{Freeman1996}.
%
Roughly speaking, one must choose control inputs such that the energy of the
system is exponentially stable through the continuous dynamics.
%
Because the CLF is affine with respect to the control input, the problem can be
posed as a quadratic program operating on a convex set which results in an
optimal controller that can be evaluated with relatively low computational
cost.
%

%In particular, given a hybrid system with an exponentially stable limit
%cycle representing a periodic behavior, the application of energy shaping
%will exponentially stabilize the energy dynamics to a specified reference
%level while maintaining exponential stability of the overall hybrid system.
% 
%The presented control strategy is applicable to a wide class of mechanical
%systems including bipedal locomotion.
% 
%Moreover, the controller involves an optimization problem which is formulated as
%a quadratic program operating on a convex set, thereby permitting the use of
%existing tools for rapidly evaluating the solution.
% 
%Numerical simulations are provided to demonstrate the application and
%usefulness of energy shaping.

The development of energy shaping approaches spans the past few decades
although energy-based approaches to analytical mechanics date back over 200
years to the work of Lagrange
% \cite{Lagrange1788}
and the principles of stability utilized were first presented by Lyapunov
% \cite{Lyapunov1992}
over 100 years ago.
% 
The ideas herein build on existing results spread throughout the literature.
%
Over the last few decades, researchers have presented control schemes which seek
to achieve periodic behaviors in dynamical systems by formulating their control
objectives in terms of the energy of a system.
%
Based on McGeer's observation that compass-gait bipeds with appropriate mass
distributions can walk down shallow slopes without actuation \cite{McGeer1990},
Spong presented controlled symmetries \cite{Spong2005} as a method for obtaining
walking on a compass-gait biped on flat ground by injecting energy in a such a
way that the shaped potential energy of the robot's gait on flat ground mimicked
the potential energy of a passive biped walking down a slope.
% 
Based on this idea, Spong later provided a controller which could shape the
total energy of a compass-gait biped and showed that the controller would
guarantee asymptotic stability of the energy dynamics to a reference level
through the continuous dynamics.
%
He later demonstrated how the ideas could be extended to non-conservative
systems \cite{Spong2007} by considering energy storage functions.
%
Similarly, the authors of this presentation have extended Spong's ideas on
controlled symmetries to three-dimensional bipeds through the use of functional
Routhian reduction \cite{Grizzle2014}.


The problem as formulated falls under a class of problems involving stability of
systems with zero dynamics.
%
In \cite{Ames2014}, a similar problem was considered in which a stabilizing
control law was constructed using control Lyapunov functions to stabilize to a
zero dynamics which exhibited hybrid invariance; that is, for initial conditions
on the intersection of the switching surface and the hybrid zero dynamics
manifold, application of the reset map will result in a state which is still on
the hybrid zero dynamics.
%
This was a key assumption underlying \cite{Ames2014} but this assumption does
not hold for energy shaping as energy is generally not invariant through impact,
though there may be pathological examples which demonstrate this property.
%
In fact, for certain conservative systems like the compass-gait biped, which
exhibits local exponential stability, energy change can only occur through
discrete transitions and so impacts actually act as a stabilizing influence.
%
%Another key difference is the assumption of stability which is necessary for the
%application of energy shaping.
%
%Whereas \cite{Ames2014} requires stability of the system for states restricted
%to the hybrid zero dynamics, this paper requires stability of the nominal system
%and does not require hybrid invariance of the zero dynamics.
%
%Thus, while the problems are somewhat similar, they also have their differences
%and are applicable to different types of systems.

\bibliographystyle{plain}
\bibliography{myrefs}

\end{document}
