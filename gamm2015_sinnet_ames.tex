%
%%%%%%%%% Do not modify anything here inside
\documentclass[english]{article}
\usepackage[a4paper]{geometry}
\geometry{verbose,tmargin=2cm,bmargin=2cm,lmargin=2cm,rmargin=2cm}

\usepackage[T1]{fontenc}
\usepackage[latin9]{inputenc}
\usepackage{amsthm}
\usepackage{amsmath}
\usepackage{amsfonts}
\usepackage{babel}

\begin{document}
%%%%%%%%%   Your contribution (Title and authors) starts from here

\title{Optimal Energy-Based Control of Hybrid Systems\\ with Applications to
  Robotic Walking}

\author{\underline{Ryan W. Sinnet} and Aaron D. Ames \\
  Texas A\&M University}%
  %\\%Affiliation of author2}


%%%%%%%%% Do not modify anything here inside
\date{}
\maketitle
\thispagestyle{empty}
%%%%%%%%%   Your contribution (Abstract content) starts from here

    This presentation discusses a method of energy shaping applicable to non-smooth
    mechanical systems.
    %
    By utilizing energy level sets in periodic behaviors, energy shaping
    alters the stability properties of a mechanical system with the goal of
    achieving more desirable robustness and convergence characteristics.
    % 
    Through the use of control Lyapunov functions, this method seeks to shape
    the energy dynamics of a system while providing a formal guarantee on stability.
    %
    In particular, given a hybrid system with an exponentially stable limit
    cycle representing a periodic behavior, the application of energy shaping
    will exponentially stabilize the energy dynamics to a specified reference
    level while maintaining exponential stability of the overall hybrid system.
    %
    The provided controller is applicable to a wide class of mechanical systems
    including bipedal locomotion.
    %
    Moreover, the controller involves an optimization problem which is
    formulated as a quadratic program operating on a convex set, thereby
    permitting the use of existing tools for rapidly evaluating the solution.
    %
    Numerical simulations are provided to demonstrate the application and
    usefulness of energy shaping.
 \cite{Sinnet2014}

\bibliographystyle{plain}
\bibliography{myrefs}

\end{document}
